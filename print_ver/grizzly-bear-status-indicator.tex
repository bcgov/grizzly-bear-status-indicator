%%%%%%%%%%%%%%%%%%%%%%%%%%%%%%%%%%%%%%%%%%%%%%%%%%%%%%%%%%%%%%%%%%%%%%%%%%%%%%%%
  % Document template setup, don't change anything here
% Requires the file C:\Local Tex Files\tex\latex\biblatex\biblatex.cfg
% for proper formatting of citations and bibliography
\documentclass[a4paper, 11pt, parskip=half]{scrartcl}

% fix for pandoc 1.14
\providecommand{\tightlist}{%
  \setlength{\itemsep}{0pt}\setlength{\parskip}{0pt}}

% Setting margins
\usepackage{geometry}
\geometry{verbose,tmargin=3.5cm,bmargin=2.5cm,lmargin=2.5cm,headsep=0.5cm,
rmargin=2.5cm,headheight=1.5cm}

% packages for headers/footers and graphics
\usepackage{graphicx}
\usepackage{grffile}
\graphicspath{{C:/Users/genperk/R/win-library/3.6/envreportutils.internal/rmarkdown/templates/print_ver/resources/img/}}
\usepackage{fancyhdr}
\pagestyle{fancy}
\fancyhf{}
\renewcommand{\headrulewidth}{0pt}
\rhead{\includegraphics[height=55pt]{MoE_EnvReportBC.pdf}}
\fancyfoot[C]{{\thepage}}

% Set font to use avantgarde (close to Century Gothic)
% \renewcommand{\familydefault}{\sfdefault} % allow sans serif fonts
% \usepackage{avant}
% \usepackage[urw-garamond]{mathdesign}
% \usepackage[T1]{fontenc}


% Parse topic names (used for png filenames) to be titlecase topic titles
\usepackage{xstring}

\newcommand{\topictitle}[1]{
  \IfStrEqCase{#1}{
      {air}{Air}
      {climate-change}{Climate Change}
      {land}{Land \& Forests}
      {plants-and-animals}{Plants \& Animals}
      {sustainability}{Sustainability}
      {water}{Water}}
      [Please choose a topic]
}

% allow landscape orientation
\usepackage{lscape}

% Caption formatting plus options
\usepackage{caption}
\captionsetup{justification=justified,singlelinecheck=off}

% Set padding in fboxes (otherwise very tight to text)
\setlength\fboxsep{0.5cm}

% Everything from default template from below the font stuff:

\usepackage{amssymb,amsmath}
\usepackage{ifxetex,ifluatex}
\usepackage{fixltx2e} % provides \textsubscript
\ifnum 0\ifxetex 1\fi\ifluatex 1\fi=0 % if pdftex
  \usepackage[T1]{fontenc}
  \usepackage[utf8]{inputenc}
\else % if luatex or xelatex
  \ifxetex
    \usepackage{mathspec}
    \usepackage{xltxtra,xunicode}
  \else
    \usepackage{fontspec}
  \fi
  \defaultfontfeatures{Mapping=tex-text,Scale=MatchLowercase}
  \newcommand{\euro}{€}
    \setmainfont{Palatino Linotype}
    \setsansfont{Calibri}
\fi
% use upquote if available, for straight quotes in verbatim environments
\IfFileExists{upquote.sty}{\usepackage{upquote}}{}
% use microtype if available
\IfFileExists{microtype.sty}{\usepackage{microtype}}{}
\usepackage{graphicx}
\makeatletter
\def\maxwidth{\ifdim\Gin@nat@width>\linewidth\linewidth\else\Gin@nat@width\fi}
\def\maxheight{\ifdim\Gin@nat@height>\textheight\textheight\else\Gin@nat@height\fi}
\makeatother
% Scale images if necessary, so that they will not overflow the page
% margins by default, and it is still possible to overwrite the defaults
% using explicit options in \includegraphics[width, height, ...]{}
\setkeys{Gin}{width=\maxwidth,height=\maxheight,keepaspectratio}
\ifxetex
  \usepackage[setpagesize=false, % page size defined by xetex
              unicode=false, % unicode breaks when used with xetex
              xetex]{hyperref}
\else
  \usepackage[unicode=true]{hyperref}
\fi
\hypersetup{breaklinks=true,
            bookmarks=true,
            pdfauthor={},
            pdftitle={Grizzly Bear Population Status Indicator},
            colorlinks=true,
            citecolor=blue,
            urlcolor=blue,
            linkcolor=blue,
            pdfborder={0 0 0}}
\urlstyle{same}  % don't use monospace font for urls
\setlength{\parindent}{0pt}
\setlength{\parskip}{6pt plus 2pt minus 1pt}
\setlength{\emergencystretch}{3em}  % prevent overfull lines
\setcounter{secnumdepth}{0}

  \title{\vspace{-1cm}
         \flushleft{\includegraphics[height=50pt]{plants-and-animals.png}
                    \hspace{0.3cm}\Huge{\topictitle{plants-and-animals}}}
         \vspace{-0.5cm}
         \begin{center}
         \line(1,0){450}
         \end{center}
  }
\subtitle{\LARGE{Grizzly Bear Population Status Indicator}}

\date{}

\begin{document}
\maketitle
\vspace{-2cm} % suppresses area where date would go... but a bit of a hack
\thispagestyle{fancy}

\begin{itemize}
\tightlist
\item
  There are approximatley XX Grizzly bears in Brish Columbia
  \textsuperscript{1}
\end{itemize}

** In BC, the grizzly bear population is divided into 55 Grizzly Bear
Population Units (GBPUs).

\begin{itemize}
\item
  Using GBPUs helps us to identify localised concenservation concerns
  and prioritise management across the province.
\item
  Each GBPU is assigned a management status (M5 to M1) which represents
  least concern to highest concern. Management status are based on
  population, isolation, trends and levels of threats. This is based on
  the NatureServe status assesment method \textsuperscript{2}
\end{itemize}

\includegraphics{grizzly-bear-status-indicator_files/figure-latex/unnamed-chunk-1-1.pdf}

\hypertarget{references-and-other-useful-links}{%
\subsection{References and Other Useful
Links}\label{references-and-other-useful-links}}

\textsuperscript{1}{[}2019 Grizzly Bear Population Unit Management
Ranking{]} (\url{http://} UPDATE THIS TO RELEASED DATA SET)

\textsuperscript{2} refence for nature Serve

\begin{itemize}
\tightlist
\item
  {[}British Columbia conservation Foundation's Bear Aware Program{]}
  (\url{https://wildsafebc.com/grizzly-bear/})
\end{itemize}

\hypertarget{data}{%
\subsection{Data}\label{data}}

*By accessing these datasets, you agree to the licence associated with
each file, as indicated in parentheses below.

\begin{itemize}
\tightlist
\item
  \href{https://catalogue.data.gov.bc.ca/dataset/bc-grizzly-bear-habitat-classification-and-rating}{Indicator
  data: BC Grizzly Bear Habitat Classification and Rating}
\end{itemize}

\hypertarget{section}{%
\subsection{\texorpdfstring{\newpage}{}}\label{section}}

Published and Available On-Line at Environmental Reporting BC (August
2019):\\
\url{http://www.env.gov.bc.ca/soe/indicators/plants-and-animals/grizzly-bears.html}

Email correspondence to:
\href{mailto:envreportbc@gov.bc.ca}{\nolinkurl{envreportbc@gov.bc.ca}}

\emph{Suggested Citation}:\\
Environmental Reporting BC. 2019. Grizzly Bear Population Status in B.C.
State of Environment Reporting, Ministry of Environment and Climate
Change Strategy, British Columbia, Canada.

\end{document}
